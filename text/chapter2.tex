\chapter{関連研究および基礎概念}

\section{SCS}
SCSは,アミノ酸配列中から一定の長さで抜き出した部分配列のことである.例えば,SCSの長さを3,配列をACDEFGHIとすると,存在するSCSはACD,CDE,DEF,EFG,FGH,GHIである.本研究では,SCSの長さを5としている.
\subsection{SCS Package}
SCS Packageは,アミノ酸断片配列(SCS)に基づくアミノ酸組利用度を用いたバイオインフォマティクス研究に有用なWebサービスである\cite{scspackage}.

%\subsection{}
% \subsection{SCSとSCS Idiom}
% SCSは,アミノ酸配列中から一定の長さで抜き出した部分配列のことである.SCS Idiomは,アミノ酸配列中のあるSCSに注目し,そのSCSの先頭を中心に,左右一定距離$D$の範囲内に存在するSCSと,注目したSCSの組み合わせのことをいう.このとき,注目したSCSを\textit{Core},もう一方を\textit{Sub}と呼ぶ.\textit{Core}と\textit{Sub}は重複部分がなく,最も近いSCS Idiomの距離$D$を1とする.図\ref{fig:scs}は,長さを3としたSCSとSCS Idiomの例である.\textit{Core}としてMSFを指定した時,SCS Idiomの\textit{Sub}としてDPI,LTY,YEVなどが存在する.
%ここに図
%\begin{figure}[htbp]
  %\begin{center}
   %\includegraphics[width=8cm]{scs.pdf}
   %\caption{長さ3のSCSとSCS Idiomの例}
   %\label{fig:scs}
   %\end{center}
%\end{figure}
\subsection{SCS Packageの概要}
 SCS Packageは,SCSに基づくタンパク質解析のためのWebサービスで,複数の機能を提供している.サービスの1つである種別配列検索システムは,用意されている種のうち検索対象にするものを1つ選択し,さらに検索したいアミノ酸の部分配列を指定することで,部分配列が含まれているその種のタンパク質の一覧を表示する.

\section{SARS-Covid-2}
\subsection{Receptor Binding Domain}
\subsection{変異株}



% \subsection{関連研究について}
% 関連研究は,自分の研究テーマの社会的もしくは学術的な意義付けを明確にし,研究テーマの主題すなわち「問い」や「アイディア」を具体化・明確化するために記述する.

% 社会的もしくは学術的な意義付けでは,「知能情報工学」の視点を持って,歴史的な経緯を時系列的に説明したり社会課題を述べたりすると良い.

% 研究テーマの主題を明らかにするために,単に類似の研究や過去の研究を説明するのではなく,それら研究が自分の主題とどのように関連しているかを具体的に述べる.自身の研究が既存の研究をより発展させたものであるならば,既存の研究では何のために何がどこまで提案・実装されているかを明確に述べ,さらにそれを発展させる必要性などを述べる.自分の研究が既存の研究を改良するものであるならば,既存の研究が何のために何をどこまで提案・実装しているかを明確に述べ,どこに問題があって,どのような解決が望まれているかなどを述べる.

% これら記述によって,次節の提案手法や考察での既存研究と自身の研究成果との比較検討の土台を作ることを目指す.

% \subsection{参考文献について}
% 参考文献を記載する目的は,他者の著作権たる知的財産権を保護するとともに,自己の主張すなわち執筆論文の責任範囲を明確にすることである.倫理講習で学んでいるように,参考文献を適切に記載しないことは論文における不正行為となる\cite{ref-jsps}.

% 参考文献とは引用した文献の総称をいう.引用には間接引用と直接引用がある.間接引用した文献を参考文献といい,参考とは他者の執筆した文章や作成した資料を自分の主張の材料とすることをいう.他方,直接引用した文献を引用文献といい,引用とは他者の文章や作成資料の一部を自分の文書の一部として含めることをいう.

% 参考にした資料,引用した資料を特定できる情報を「出典」といい,論文,書籍,新聞,Webサイトの情報などが出典にあたる.出典は,本文書の末尾にthebibliography環境を使用してリストし,cite命令を使用して本文中で引用する.なお,出典の表示項目と表示例については,各自の研究分野の慣例に従うか,電子情報通信学会や情報処理学会の文献記載例に従うものとする\cite{ref-ieice,ref-ipsj}.例えば,電子情報通信学会の文献記載例では,学術論文(article, journal)の場合\cite{ref-journal,ref-journal-ex},著作の場合\cite{ref-book,ref-book-ex},学会論文(proceedings)の場合\cite{ref-proceedings,ref-proceedings-ex},Webページの場合\cite{ref-web, ref-ieice}のようになる.

.

