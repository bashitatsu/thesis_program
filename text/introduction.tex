\chapter{序論}
%\thispagestyle{fancy}
本章は,標準として「背景と目的」および「論文の構成」の2節で構成される.分量の規定はないが,2節合わせて2$\sim $3ページ程度が標準的である.

\section{背景と目的}
この節には,研究の背景と目的を記述する.研究の社会的もしくは学術的な意義を明らかにするものとする.

\section{論文の構成}
この節では,卒業論文の全体構成を述べる.論文の構成は研究分野によって異なるので,指導教員と調整すると良い.その上で,卒業論文の内容に従って以下の例のように論文の構成を記述しなさい.この記述形式でないといけないということではないので,各自で適切に調整しなさい.

本論文は以下の章で構成される.第1章の○○(○○に章タイトルを記述する)に続き,第2章の関連研究では△△△について述べる.第3章では,□□□について本論文で新たに提案する手法について述べる.第4章では,提案手法での植物形態の三次元再構成精度検証実験について述べるとともに,既存の研究結果との比較考察を述べる.最後に,第5章で本論文のまとめと今後の展望について述べる.
